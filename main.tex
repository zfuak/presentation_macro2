\documentclass[10pt]{beamer}

\usetheme[progressbar=frametitle]{metropolis}
\usepackage{appendixnumberbeamer}

\usepackage{booktabs}
\usepackage[scale=2]{ccicons}

\usepackage{pgfplots}
\usepgfplotslibrary{dateplot}

\usepackage{xspace}
\newcommand{\themename}{\textbf{\textsc{metropolis}}\xspace}

\title{Output Dynamics in Real-Business-Cycle Models}
\subtitle{Timothy Cogley and James M. Nason}
% \date{\today}
\date{}
\author{Zixuan, Huixin, Rémi}
\institute{M2 ETE Macroeconomics 2}
% \titlegraphic{%
%     \includegraphics[width=.3\textwidth]{TSE_Logo_2019.png}\hfill
% }

% \makeatletter
% \setbeamertemplate{title page}{
%   \begin{minipage}[b][\paperheight]{\textwidth}
%     \vfill%
%     \ifx\inserttitle\@empty\else\usebeamertemplate*{title}\fi
%     \ifx\insertsubtitle\@empty\else\usebeamertemplate*{subtitle}\fi
%     \usebeamertemplate*{title separator}
%     \ifx\beamer@shortauthor\@empty\else\usebeamertemplate*{author}\fi
%     \ifx\insertdate\@empty\else\usebeamertemplate*{date}\fi
%     \ifx\insertinstitute\@empty\else\usebeamertemplate*{institute}\fi
%     \vfill
%     \ifx\inserttitlegraphic\@empty\else\inserttitlegraphic\fi
%     \vspace*{1cm}
%   \end{minipage}
% }
% \makeatother

\begin{document}

\maketitle

\begin{frame}{Table of contents}
  \setbeamertemplate{section in toc}[sections numbered]
  \tableofcontents%[hideallsubsections]
\end{frame}

\section{Introduction}

\begin{frame}{Introduction}
    It establishes a link between the \alert{theoretical} RBC model literature and the \alert{empirical} output dynamics analysis.
    \begin{itemize}
        \item Theoretical (3 models): 
        \begin{enumerate}
            \item Baseline RBC
            \item Baseline + capital adjustment cost
            \item Baseline + labor adjustment cost
        \end{enumerate}
        \item Empirical (4 graphs)
        \begin{enumerate}
            \item output growth $\Delta y_t$ behavior: autocorrelation function \& spectrum decomposition
            \item impulse response: to \alert{permanent} shock $\varepsilon_z$ \& \alert{temporary} shock $\varepsilon_g$
        \end{enumerate} 
    \end{itemize}
They calibrate the model with real world data $\to$ simulate shocks to feed into the model $\to$ graph and compare the simulated responses with that of empirical observations.

    \note[itemize]{
    \item point 1
    \item point 2
    }
\end{frame}
\begin{frame}{Focus (all about shocks...)}
\metroset{block=fill}
        \begin{exampleblock}{Theory}
            Propagation mechanisms embedded in the model resulted in different $\Delta y_t$ responses to shocks $\varepsilon_z$ and $\varepsilon_g$ 
        \end{exampleblock}
        \begin{align}
            &(1-L) \ln \left(a_{t}\right)=\mu+\varepsilon_{\mathrm{a} t} \\
            &\ln \left(g_{t}\right)-\ln \left(a_{t}\right)=\bar{g}+\varepsilon_{g t} /(1-\rho L) 
        \end{align}
     
        \begin{exampleblock}{Empirical}
        An important component in empirical analysis is how to identify $\varepsilon_z$ and $\varepsilon_g$ from observation of $\Delta y_t$.
    \end{exampleblock}
\end{frame}


\section{Main Idea}
\subsection{Baseline Model}

\begin{frame}{Estimation and Simulation}
    \begin{itemize}
        \item Estimation
            \begin{itemize}
                \item Christiano and Eichenbaum (1992) estimate, $\mu = 0.004, \bar{g} = 0.177$, and $p = 0.96$
                \item Rescale the innovation variances to match the sample variance of per capita GNP growth: $\sigma_a = 0.0097$ and $\sigma_g = 0.0113$
            \end{itemize}
        \item Monte Carlo Simulation
            \begin{itemize}
                \item Generate artificial data over a time horiwzon of 140 quarters to match the length of sample period
                \item Each model was simulated 1,000 times
                \item Autocorrelation and impulse-response functions were estimated for each artificial sample
            \end{itemize}
        
    \end{itemize}
\end{frame}


\begin{frame}{Baseline model simulation and comparison}
\begin{figure}
    \centering
  \includegraphics[width=0.5\linewidth]{baseline_all.png}
  %\caption{}
\end{figure}
\begin{itemize}
    \item  ACF are close to zero, output growth is close to being white noise $\rightarrow$  No serial correlation
    \item Spectrum is quite flat $\rightarrow$ No Business-cycle periodicity in output growth
    \item The model strongly damps transitory shock and generates monotonic decay $\rightarrow$ No trend-reverting component in output
\end{itemize}
\end{frame}

\subsection{Gestation Lags and Capital Adjustment Costs}

\begin{frame}{Gestation Lags and Capital Adjustment Costs}
    Production function becomes:
    $$
\ln \left(y_t\right)= \ln \left[f\left(k_t, a_t n_t\right)\right] -\left(\alpha_{k} / 2\right)\left[\Delta k_t / k_{t-1}\right]^2
$$

Based on Shapiro (1986), $\alpha_{k}$ is calibrated to be 2.2

\end{frame}

\begin{frame}{Comparison}
\begin{figure}
    \centering
  \includegraphics[width=0.6\linewidth]{Capital adj cost.png}
  %\caption{}
\end{figure}
\begin{itemize}
    \item No serial correlation or Business-cycle periodicity in output growth
    \item No effect on IRF $\rightarrow$ No help to propagate shocks
    \begin{itemize}
        \item Gestation Lags and Capital Adjustment Costs alter $I_t$, but $I_t$ is small relative to $k_t$ $\rightarrow$ little effect on $y_t$
    \end{itemize}
\end{itemize}
    
\end{frame}

\subsection{Employment Lags and Labor Adjustment Costs}
\begin{frame}{Employment Lags and Labor Adjustment Costs}
\textbf{Adjustment-cost model}

Production function:
$$
\ln \left(y_t\right)= \ln \left[f\left(k_t, a_t n_t\right)\right] -\left(\alpha_{k} / 2\right)\left[\Delta k_t / k_{t-1}\right]^2 -\left(\alpha_n / 2\right)\left[\Delta n_t / n_{t-1}\right]^2
$$

Shapiro's estimate : $\alpha_{k} = 0.36$


\textbf{Labor-hoarding model} of Burnside et al. (1993): firms must choose the size of the labor force before observing the current state of the economy but can vary the intensity of work effort after observing the current state

\end{frame}

\begin{frame}{Comparison}
\begin{figure}
    \centering
  \includegraphics[width=0.6\linewidth]{labor cost.png}
  %\caption{}
\end{figure}
\begin{itemize}
    \item Positive autocorrelation in output growth
    \begin{itemize}
        \item Burnside et al. (1993) model: positively autocorrelated at lag 1 and has modest negative autocorrelation at higher-order lags
        \item Adjustment-cost model: output growth is well approximated by an AR(1), with positive autocorrelation at lag 1 and monotonic decay at higher-order lags
    \item Modest serial correlation
    \end{itemize}
\end{itemize}
    
\end{frame}



\end{document}
